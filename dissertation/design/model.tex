\section{Reactive Component Model\label{model}}

\paragraph{Reactive components.}
A reactive component consists of \emph{state}, \emph{actions}, \emph{reactions}, \emph{ports}, \emph{sub-components}, \emph{bindings}, and \emph{accessors}.
The state of a component may be finite or logically infinite.
An action consists of a \emph{precondition} and \emph{effect}.
A precondition is a Boolean expression that indicates that the action can be executed.
The effect specifies a deterministic update to the state variables.
A reaction is a named effect.
A port is a named interaction point used to link an action or reaction to a reaction.
A reactive component may contain sub-components.
A binding links a port to a reaction.
An accessor synchronously reads the state of a component.

\paragraph{Effect sets.}
Effects can conditionally \emph{trigger} ports.
If a port is bound to a reaction, then the reaction is applied with the triggering effect.
Let $(c,e)$ be a pair consisting of the component $c$ and the effect $e$.
We can define the relation $(c_0, e_0) triggers (c_1, e_1)$.
The transitive closure $Triggers(c,e)$ forms the \emph{effect set} of $(c,e)$.

\paragraph{Execution.}
Computation proceeds by selecting a component-action pair $(c,e)$, evaluating the precondition, and applying the effect and all triggered effects in the effect set $Triggers(c,e)$ if the precondition is true.
Actions are selected non-deterministically and their corresponding effect sets are executed atomically with respect to each other.
These tasks are performed by a \emph{scheduler}.

\paragraph{Non-interference condition.}
The state of a reactive component can only be updated by its associated actions and reactions.

\paragraph{Deterministic execution condition.}
The execution of each action must result in a unique next state, that is, the execution of each action must be deterministic.

\paragraph{Explanation of conditions.}
The ultimate goal of both conditions is to facilitate the development of robust, predictable systems.
The conditions guarantee certain behaviors which in turn allows programmers to reason about the systems they build.
Non-interference allows a component developer to establish the properties (safety, progress) of a component from the text of the component.
This is analogous to information hiding in OOP.
The deterministic execution condition prevents a component from interfering with itself when composed.
Let us consider a slightly more complex model of execution.

Let each effect consists of an \emph{immutable phase} and a \emph{mutable phase}.
A component can perform arbitrary computation in the immutable phase (including accessing the state of other components with accessors) but cannot update its own state.
In the mutable phase, state updates are allowed but accessors are not.
Ports are triggered in between the two phases.
Thus, when applying an effect set, all immutable phases are processed, ports are triggered, and then all of the mutable phases are processed.

For deterministic execution, we only care about the mutable phase.
An effect is \emph{potent} if its mutable phase might perform a state update.
For deterministic execution, we require that there is at most one potent effect for a component in an effect set.
Furthermore, we require that the $triggers$ relation induce a tree structure as opposed to an arbitrary directed graph.
A effect triggered by multiple ports (including cycles) may require that the state be updated in conflicting ways, hence, the structure is restricted to a tree.

I believe the formulation above is more tractable than a more general formulation that involves looking at the individual variables updated by the effect.
For example, we can consider a component with two state variables and corresponding reactions that update the state variables.
It is obvious that both reactions can be in the same effect set since the state on which they operate is disjoint.
However, the situation becomes much more difficult if the state variables represent arbitrary linked data structures.
The deterministic execution condition would then require pointer analysis which is is difficult.
I hypothesize the existence of a decomposition theorem that allows state variables or groups of state variables to be replaced with sub-components which could then be analyzed using the ``single potent effect'' framework.

%% This requirement can be interpretted for individual action trees.
%% Given an action tree, we can enumerate the components and by extension the state variables.
%% We can divide the set of variables into those that are read and those that are written.
%% The requirement says that each written variable must have a unique value.
%% An action tree is \emph{non-deterministic} if it violates this requirement.
%% A feedback loop occurs when the same component appears multiple times in an action tree.
%% One foreseeable way to create a non-deterministic action tree is to introduce a feedback loop that writes the same variable two different ways.

The non-deterministic condition also provides a way to loosen the assumption that the execution of each action be physically atomic.
Consider all of the components implied by two actions.
We can annotate each component as being read-only or read-write in the first action and read-only or read-write in the second action.
The two effects sets are \emph{independent} if the read-write components in each action are disjoint and the read-only components in the first action are disjoint from read-write components in the second action and vice versa.
Independent actions can be executed in parallel.

\paragraph{Summary of conditions.}
The two conditions that must be met in a system for reactive components are determinism and non-interference.
Determinism says that each action yield a unique next state.
Non-interference says that the state of a component will not be changed by any other party except the component.
The conditions must always hold but can be established at any time between compilation and execution.
These conditions are the driving force behind many of the design decisions.
